\documentclass[]{article}
\usepackage{lmodern}
\usepackage{amssymb,amsmath}
\usepackage{ifxetex,ifluatex}
\usepackage{fixltx2e} % provides \textsubscript
\ifnum 0\ifxetex 1\fi\ifluatex 1\fi=0 % if pdftex
  \usepackage[T1]{fontenc}
  \usepackage[utf8]{inputenc}
\else % if luatex or xelatex
  \ifxetex
    \usepackage{mathspec}
  \else
    \usepackage{fontspec}
  \fi
  \defaultfontfeatures{Ligatures=TeX,Scale=MatchLowercase}
\fi
% use upquote if available, for straight quotes in verbatim environments
\IfFileExists{upquote.sty}{\usepackage{upquote}}{}
% use microtype if available
\IfFileExists{microtype.sty}{%
\usepackage{microtype}
\UseMicrotypeSet[protrusion]{basicmath} % disable protrusion for tt fonts
}{}
\usepackage[margin=1in]{geometry}
\usepackage{hyperref}
\hypersetup{unicode=true,
            pdftitle={Intégration de codes python dans R avec le package reticulate},
            pdfauthor={Marie Morvan, Marine Marjou, Claire Gayral},
            pdfborder={0 0 0},
            breaklinks=true}
\urlstyle{same}  % don't use monospace font for urls
\usepackage{color}
\usepackage{fancyvrb}
\newcommand{\VerbBar}{|}
\newcommand{\VERB}{\Verb[commandchars=\\\{\}]}
\DefineVerbatimEnvironment{Highlighting}{Verbatim}{commandchars=\\\{\}}
% Add ',fontsize=\small' for more characters per line
\usepackage{framed}
\definecolor{shadecolor}{RGB}{248,248,248}
\newenvironment{Shaded}{\begin{snugshade}}{\end{snugshade}}
\newcommand{\AlertTok}[1]{\textcolor[rgb]{0.94,0.16,0.16}{#1}}
\newcommand{\AnnotationTok}[1]{\textcolor[rgb]{0.56,0.35,0.01}{\textbf{\textit{#1}}}}
\newcommand{\AttributeTok}[1]{\textcolor[rgb]{0.77,0.63,0.00}{#1}}
\newcommand{\BaseNTok}[1]{\textcolor[rgb]{0.00,0.00,0.81}{#1}}
\newcommand{\BuiltInTok}[1]{#1}
\newcommand{\CharTok}[1]{\textcolor[rgb]{0.31,0.60,0.02}{#1}}
\newcommand{\CommentTok}[1]{\textcolor[rgb]{0.56,0.35,0.01}{\textit{#1}}}
\newcommand{\CommentVarTok}[1]{\textcolor[rgb]{0.56,0.35,0.01}{\textbf{\textit{#1}}}}
\newcommand{\ConstantTok}[1]{\textcolor[rgb]{0.00,0.00,0.00}{#1}}
\newcommand{\ControlFlowTok}[1]{\textcolor[rgb]{0.13,0.29,0.53}{\textbf{#1}}}
\newcommand{\DataTypeTok}[1]{\textcolor[rgb]{0.13,0.29,0.53}{#1}}
\newcommand{\DecValTok}[1]{\textcolor[rgb]{0.00,0.00,0.81}{#1}}
\newcommand{\DocumentationTok}[1]{\textcolor[rgb]{0.56,0.35,0.01}{\textbf{\textit{#1}}}}
\newcommand{\ErrorTok}[1]{\textcolor[rgb]{0.64,0.00,0.00}{\textbf{#1}}}
\newcommand{\ExtensionTok}[1]{#1}
\newcommand{\FloatTok}[1]{\textcolor[rgb]{0.00,0.00,0.81}{#1}}
\newcommand{\FunctionTok}[1]{\textcolor[rgb]{0.00,0.00,0.00}{#1}}
\newcommand{\ImportTok}[1]{#1}
\newcommand{\InformationTok}[1]{\textcolor[rgb]{0.56,0.35,0.01}{\textbf{\textit{#1}}}}
\newcommand{\KeywordTok}[1]{\textcolor[rgb]{0.13,0.29,0.53}{\textbf{#1}}}
\newcommand{\NormalTok}[1]{#1}
\newcommand{\OperatorTok}[1]{\textcolor[rgb]{0.81,0.36,0.00}{\textbf{#1}}}
\newcommand{\OtherTok}[1]{\textcolor[rgb]{0.56,0.35,0.01}{#1}}
\newcommand{\PreprocessorTok}[1]{\textcolor[rgb]{0.56,0.35,0.01}{\textit{#1}}}
\newcommand{\RegionMarkerTok}[1]{#1}
\newcommand{\SpecialCharTok}[1]{\textcolor[rgb]{0.00,0.00,0.00}{#1}}
\newcommand{\SpecialStringTok}[1]{\textcolor[rgb]{0.31,0.60,0.02}{#1}}
\newcommand{\StringTok}[1]{\textcolor[rgb]{0.31,0.60,0.02}{#1}}
\newcommand{\VariableTok}[1]{\textcolor[rgb]{0.00,0.00,0.00}{#1}}
\newcommand{\VerbatimStringTok}[1]{\textcolor[rgb]{0.31,0.60,0.02}{#1}}
\newcommand{\WarningTok}[1]{\textcolor[rgb]{0.56,0.35,0.01}{\textbf{\textit{#1}}}}
\usepackage{graphicx,grffile}
\makeatletter
\def\maxwidth{\ifdim\Gin@nat@width>\linewidth\linewidth\else\Gin@nat@width\fi}
\def\maxheight{\ifdim\Gin@nat@height>\textheight\textheight\else\Gin@nat@height\fi}
\makeatother
% Scale images if necessary, so that they will not overflow the page
% margins by default, and it is still possible to overwrite the defaults
% using explicit options in \includegraphics[width, height, ...]{}
\setkeys{Gin}{width=\maxwidth,height=\maxheight,keepaspectratio}
\IfFileExists{parskip.sty}{%
\usepackage{parskip}
}{% else
\setlength{\parindent}{0pt}
\setlength{\parskip}{6pt plus 2pt minus 1pt}
}
\setlength{\emergencystretch}{3em}  % prevent overfull lines
\providecommand{\tightlist}{%
  \setlength{\itemsep}{0pt}\setlength{\parskip}{0pt}}
\setcounter{secnumdepth}{0}
% Redefines (sub)paragraphs to behave more like sections
\ifx\paragraph\undefined\else
\let\oldparagraph\paragraph
\renewcommand{\paragraph}[1]{\oldparagraph{#1}\mbox{}}
\fi
\ifx\subparagraph\undefined\else
\let\oldsubparagraph\subparagraph
\renewcommand{\subparagraph}[1]{\oldsubparagraph{#1}\mbox{}}
\fi

%%% Use protect on footnotes to avoid problems with footnotes in titles
\let\rmarkdownfootnote\footnote%
\def\footnote{\protect\rmarkdownfootnote}

%%% Change title format to be more compact
\usepackage{titling}

% Create subtitle command for use in maketitle
\providecommand{\subtitle}[1]{
  \posttitle{
    \begin{center}\large#1\end{center}
    }
}

\setlength{\droptitle}{-2em}

  \title{Intégration de codes python dans R avec le package reticulate}
    \pretitle{\vspace{\droptitle}\centering\huge}
  \posttitle{\par}
    \author{Marie Morvan, Marine Marjou, Claire Gayral}
    \preauthor{\centering\large\emph}
  \postauthor{\par}
      \predate{\centering\large\emph}
  \postdate{\par}
    \date{27/08/2019}

\usepackage{booktabs}
\usepackage{longtable}
\usepackage{array}
\usepackage{multirow}
\usepackage{wrapfig}
\usepackage{float}
\usepackage{colortbl}
\usepackage{pdflscape}
\usepackage{tabu}
\usepackage{threeparttable}
\usepackage{threeparttablex}
\usepackage[normalem]{ulem}
\usepackage{makecell}
\usepackage{xcolor}

\begin{document}
\maketitle

Nous allons vous présenter brièvement comment utiliser la bibliothèque
Reticulate, pour lancer un code python dans un environnement R, puis
nous donnerons deux exemple d'utilisation.
\texttt{:\ l\textquotesingle{}ouverture\ de\ fichiers\ et\ la\ comparaison\ de\ méthodes\ de\ RandomForest.}

\hypertarget{comment-ca-marche}{%
\section{1. Comment ça marche ?}\label{comment-ca-marche}}

Le package ``reticulate'' permet de lier R et Python en utilisant des
appels Python dans R. Commençons par faire appel à la bibliothèque et
donner à le ``path'' de python (c'est à dire le chemin où est installé
la version de python)

\begin{Shaded}
\begin{Highlighting}[]
\KeywordTok{library}\NormalTok{(reticulate)}
\KeywordTok{use\_python}\NormalTok{(}\StringTok{"/home/claire/Applications/python/conda/bin/python"}\NormalTok{)}
\CommentTok{\# on prend un jeu de donnée jouet : }
\KeywordTok{data}\NormalTok{(cars)}
\end{Highlighting}
\end{Shaded}

On peut également choisir l'environnement python où l'on va travailler :

\begin{Shaded}
\begin{Highlighting}[]
\KeywordTok{use\_virtualenv}\NormalTok{(}\StringTok{"myenv"}\NormalTok{)}
\end{Highlighting}
\end{Shaded}

\hypertarget{appel-de-librairies-python-dans-lenvironnement-r}{%
\subsection{1.1 Appel de librairies python dans l'environnement
R}\label{appel-de-librairies-python-dans-lenvironnement-r}}

Une première façon d'intégrer du code python dans R est d'utiliser ses
bibliothèques/ modules de travail (qui contiennent les fonctions
utilisées) avec la fonction reticulate::import(``my\_py\_library'') Par
exemple si on veut importer la bibliothèque python ``numpy'' :

\begin{Shaded}
\begin{Highlighting}[]
\NormalTok{np  =}\StringTok{ }\KeywordTok{import}\NormalTok{(}\StringTok{"numpy"}\NormalTok{)}
\end{Highlighting}
\end{Shaded}

Ensuite, on lance directement les fonctions pythons en appelant la
bibliothèque utilisée avec ``\$''. Par exemple, si on veut faire la
moyenne par variable, en faisant appel à la fonction ``mean'' de numpy
(qui existe sous l'alias np), de notre jeu de données cars :

\begin{Shaded}
\begin{Highlighting}[]
\NormalTok{np}\OperatorTok{$}\KeywordTok{mean}\NormalTok{(cars)}
\end{Highlighting}
\end{Shaded}

\begin{verbatim}
## speed  dist 
## 15.40 42.98
\end{verbatim}

Ainsi, on peut directement appeler les fonctions des différents packets.
Cependant attention, les types de variable python et R sont traduits par
``reticulate'' selon une table de conversion bien précise. Ca va être le
gros soucis du passage d'un langage à l'autre.

\hypertarget{fenetre-interactive}{%
\subsection{1.2 Fenêtre interactive}\label{fenetre-interactive}}

Un deuxième moyen d'utiliser du python dans un script R est d'ouvrir une
fenêtre intéractive avec repl\_python :

\begin{Shaded}
\begin{Highlighting}[]
\KeywordTok{repl\_python}\NormalTok{()}
\NormalTok{from sklearn.ensemble import RandomForestClassifier}
\NormalTok{clf =}\StringTok{ }\KeywordTok{RandomForestClassifier}\NormalTok{(}\DataTypeTok{n\_estimators=}\DecValTok{20}\NormalTok{)}
\NormalTok{exit}
\end{Highlighting}
\end{Shaded}

Le problème est que ça ne marche pas dans un markedown. Cette approche a
l'air conçue pour une utilisation dans un terminal.

\hypertarget{appel-dun-script-python}{%
\subsection{1.3 Appel d'un script
python}\label{appel-dun-script-python}}

La dernière façon est de charger un script `.py' dans le script R/Rmd :
les fonctions sont alors directement appelée comme dans 1.1, et on n'a
le problème de typage que sur les variables d'entrées et de sortie (on
reste dans le langage python. Pour lancer par exemple le script
``test.py'', la commande est la suivante :

\begin{Shaded}
\begin{Highlighting}[]
\KeywordTok{use\_python}\NormalTok{(}\StringTok{"/home/claire/Applications/python/conda/bin/python"}\NormalTok{)}
\KeywordTok{source\_python}\NormalTok{(}\StringTok{\textquotesingle{}test.py\textquotesingle{}}\NormalTok{)}
\KeywordTok{my\_mean}\NormalTok{(cars)}
\end{Highlighting}
\end{Shaded}

\begin{verbatim}
## speed  dist 
## 15.40 42.98
\end{verbatim}

Le script ``test.py'' contient les fonctions (python) suivantes :

\hypertarget{probleme-de-typage}{%
\subsection{1.4 Problème de typage :}\label{probleme-de-typage}}

Comme évoqué précédemment, le gros problème de la bibliothèque est la
conversion entre les types python et les types R. La table de conversion
est la suivante :

\[
\begin{tabular}{c c}
\hline X & P(X = i) \T \\\hline
  1 \T & 1/6 \\\hline
  2 \T & 1/6 \\\hline
  3 \T & 1/6 \\\hline
  4 \T & 1/6 \\\hline
  5 \T & 1/6 \\\hline
  6 \T & 1/6 \\\hline
\end{tabular}\]

Par exemple, * utiliser le script test pour montrer des conversion un
peu bizarre (type int par ex)*

Par ailleurs, l'appel de modules python dans un script R nécessite
souvent des transformations de type : un array n'est pas un vector, etc.

\hypertarget{un-premier-cas-detude-ouverture-de-fichiers-exotiques}{%
\section{2. Un premier cas d'étude : ouverture de fichiers
exotiques}\label{un-premier-cas-detude-ouverture-de-fichiers-exotiques}}

Nous allons présenter comment utiliser la bibliothèque ``reticulate'',
et mettre en évidence ses limites, au travers de l'ouverture de fichier
via la librairie pandas.

\hypertarget{ouverture-de-csv-classiques}{%
\subsection{2.1 Ouverture de csv
``classiques''}\label{ouverture-de-csv-classiques}}

\hypertarget{ouverture-de-gros-csv}{%
\subsection{2.2 Ouverture de gros csv}\label{ouverture-de-gros-csv}}

\hypertarget{ouverture-de-json-emboites}{%
\subsection{2.3 Ouverture de json
emboités}\label{ouverture-de-json-emboites}}

\hypertarget{comparaison-des-resultats-de-random-forest-entre-r-et-python-via-r}{%
\section{3. Comparaison des résultats de Random Forest entre R et python
via R
:}\label{comparaison-des-resultats-de-random-forest-entre-r-et-python-via-r}}

\hypertarget{comparaison-des-resultats-de-randomforest-issus-de-plusieurs-fonctions-r}{%
\subsection{3.1 Comparaison des résultats de RandomForest issus de
plusieurs fonctions
R}\label{comparaison-des-resultats-de-randomforest-issus-de-plusieurs-fonctions-r}}

\begin{Shaded}
\begin{Highlighting}[]
\KeywordTok{library}\NormalTok{(parsnip)}
\KeywordTok{library}\NormalTok{(mlbench) }
\KeywordTok{library}\NormalTok{(randomForest)}
\KeywordTok{library}\NormalTok{(ranger)}
\end{Highlighting}
\end{Shaded}

Le package mlbench met a disposition des données de machine learning. On
utilise les données DNA comprenant 3 186 observations de 180 variables,
et la variable de classe à prédire.

\begin{Shaded}
\begin{Highlighting}[]
\KeywordTok{data}\NormalTok{(DNA)}
\NormalTok{train =}\StringTok{ }\KeywordTok{sample}\NormalTok{(}\DecValTok{1}\OperatorTok{:}\KeywordTok{nrow}\NormalTok{(DNA), }\DataTypeTok{size =} \KeywordTok{nrow}\NormalTok{(DNA)}\OperatorTok{/}\DecValTok{2}\NormalTok{)}
\NormalTok{test  =}\StringTok{ }\KeywordTok{setdiff}\NormalTok{(}\DecValTok{1}\OperatorTok{:}\KeywordTok{nrow}\NormalTok{(DNA), train)}
\end{Highlighting}
\end{Shaded}

\hypertarget{methode-classique}{%
\subsubsection{3.1.1 Méthode classique}\label{methode-classique}}

On utilise directement les différentes fonctions permettant de faire des
forêts aléatoires.

\begin{Shaded}
\begin{Highlighting}[]
\NormalTok{t0rF =}\StringTok{ }\KeywordTok{Sys.time}\NormalTok{() }
\NormalTok{RFrF =}\StringTok{ }\KeywordTok{randomForest}\NormalTok{(Class }\OperatorTok{\textasciitilde{}}\StringTok{ }\NormalTok{., }\DataTypeTok{data =}\NormalTok{ DNA[train, ])}
\NormalTok{t1rF =}\StringTok{ }\KeywordTok{Sys.time}\NormalTok{() }
\NormalTok{tcalc\_rF =}\StringTok{ }\NormalTok{t1rF }\OperatorTok{{-}}\StringTok{ }\NormalTok{t0rF}

\NormalTok{t0r =}\StringTok{ }\KeywordTok{Sys.time}\NormalTok{() }
\NormalTok{RFr =}\StringTok{ }\KeywordTok{ranger}\NormalTok{(Class }\OperatorTok{\textasciitilde{}}\StringTok{ }\NormalTok{., }\DataTypeTok{data =}\NormalTok{ DNA[train, ])}
\NormalTok{t1r =}\StringTok{ }\KeywordTok{Sys.time}\NormalTok{() }
\NormalTok{tcalc\_r =}\StringTok{ }\NormalTok{t1r }\OperatorTok{{-}}\StringTok{ }\NormalTok{t0r}

\NormalTok{tcalc\_rF}
\end{Highlighting}
\end{Shaded}

\begin{verbatim}
## Time difference of 1.215034 secs
\end{verbatim}

\begin{Shaded}
\begin{Highlighting}[]
\NormalTok{tcalc\_r}
\end{Highlighting}
\end{Shaded}

\begin{verbatim}
## Time difference of 0.5531094 secs
\end{verbatim}

\begin{Shaded}
\begin{Highlighting}[]
\NormalTok{RFrF}
\end{Highlighting}
\end{Shaded}

\begin{verbatim}
## 
## Call:
##  randomForest(formula = Class ~ ., data = DNA[train, ]) 
##                Type of random forest: classification
##                      Number of trees: 500
## No. of variables tried at each split: 13
## 
##         OOB estimate of  error rate: 4.27%
## Confusion matrix:
##     ei  ie   n class.error
## ei 371  10  15  0.06313131
## ie  12 329  12  0.06798867
## n    8  11 825  0.02251185
\end{verbatim}

\begin{Shaded}
\begin{Highlighting}[]
\NormalTok{RFr}
\end{Highlighting}
\end{Shaded}

\begin{verbatim}
## Ranger result
## 
## Call:
##  ranger(Class ~ ., data = DNA[train, ]) 
## 
## Type:                             Classification 
## Number of trees:                  500 
## Sample size:                      1593 
## Number of independent variables:  180 
## Mtry:                             13 
## Target node size:                 1 
## Variable importance mode:         none 
## Splitrule:                        gini 
## OOB prediction error:             4.27 %
\end{verbatim}

\hypertarget{avec-parsnip}{%
\subsubsection{3.1.2 Avec parsnip}\label{avec-parsnip}}

Le package permet d'homogeneiser l'utilisation de différentes fonctions
permettant de faire de la classification par forêts aléatoires, et de
faciliter la comparaison des résultats obtenus.

\begin{Shaded}
\begin{Highlighting}[]
\NormalTok{model =}\StringTok{ }\KeywordTok{rand\_forest}\NormalTok{(}\DataTypeTok{mode =} \StringTok{"classification"}\NormalTok{, }\DataTypeTok{trees =} \DecValTok{2000}\NormalTok{)}

\NormalTok{rf\_randomForest <{-}}\StringTok{ }
\StringTok{  }\NormalTok{model }\OperatorTok{\%>\%}
\StringTok{  }\KeywordTok{set\_engine}\NormalTok{(}\StringTok{"randomForest"}\NormalTok{) }\OperatorTok{\%>\%}
\StringTok{  }\KeywordTok{fit}\NormalTok{(Class }\OperatorTok{\textasciitilde{}}\StringTok{ }\NormalTok{., }\DataTypeTok{data =}\NormalTok{ DNA[train, ]) }

\NormalTok{pred\_rF =}\StringTok{ }\KeywordTok{predict}\NormalTok{(rf\_randomForest, DNA[test, ])}

\NormalTok{rf\_ranger <{-}}
\StringTok{  }\NormalTok{model }\OperatorTok{\%>\%}
\StringTok{  }\KeywordTok{set\_engine}\NormalTok{(}\StringTok{"ranger"}\NormalTok{) }\OperatorTok{\%>\%}
\StringTok{  }\KeywordTok{fit}\NormalTok{(Class }\OperatorTok{\textasciitilde{}}\StringTok{ }\NormalTok{., }\DataTypeTok{data =}\NormalTok{ DNA[train, ])}

\NormalTok{pred\_r =}\StringTok{ }\KeywordTok{predict}\NormalTok{(rf\_ranger, DNA[test, ])}

\KeywordTok{library}\NormalTok{(mclust)}
\end{Highlighting}
\end{Shaded}

\begin{verbatim}
## Package 'mclust' version 5.4.5
## Type 'citation("mclust")' for citing this R package in publications.
\end{verbatim}

\begin{Shaded}
\begin{Highlighting}[]
\KeywordTok{adjustedRandIndex}\NormalTok{(DNA}\OperatorTok{$}\NormalTok{Class[test], pred\_rF}\OperatorTok{$}\NormalTok{.pred\_class)}
\end{Highlighting}
\end{Shaded}

\begin{verbatim}
## [1] 0.8617965
\end{verbatim}

\begin{Shaded}
\begin{Highlighting}[]
\KeywordTok{adjustedRandIndex}\NormalTok{(DNA}\OperatorTok{$}\NormalTok{Class[test], pred\_r}\OperatorTok{$}\NormalTok{.pred\_class)}
\end{Highlighting}
\end{Shaded}

\begin{verbatim}
## [1] 0.8614648
\end{verbatim}

\begin{Shaded}
\begin{Highlighting}[]
\KeywordTok{sum}\NormalTok{(DNA}\OperatorTok{$}\NormalTok{Class[test] }\OperatorTok{!=}\StringTok{ }\NormalTok{pred\_rF}\OperatorTok{$}\NormalTok{.pred\_class)}\OperatorTok{/}\KeywordTok{length}\NormalTok{(test)}\OperatorTok{*}\DecValTok{100}
\end{Highlighting}
\end{Shaded}

\begin{verbatim}
## [1] 4.959196
\end{verbatim}

\begin{Shaded}
\begin{Highlighting}[]
\KeywordTok{sum}\NormalTok{(DNA}\OperatorTok{$}\NormalTok{Class[test] }\OperatorTok{!=}\StringTok{ }\NormalTok{pred\_r}\OperatorTok{$}\NormalTok{.pred\_class)}\OperatorTok{/}\KeywordTok{length}\NormalTok{(test)}\OperatorTok{*}\DecValTok{100}
\end{Highlighting}
\end{Shaded}

\begin{verbatim}
## [1] 5.021971
\end{verbatim}

\hypertarget{avec-python}{%
\subsection{3.2 Avec Python}\label{avec-python}}

\hypertarget{en-utilisant-directement-les-modules-python-pour-modeliser-les-donnees}{%
\subsection{3.2.1 En utilisant directement les modules python pour
modéliser les
données}\label{en-utilisant-directement-les-modules-python-pour-modeliser-les-donnees}}

Importer les modules de travail (qui contiennent les fonctions
utilisées)

\begin{Shaded}
\begin{Highlighting}[]
\NormalTok{np  =}\StringTok{ }\KeywordTok{import}\NormalTok{(}\StringTok{"numpy"}\NormalTok{)}
\NormalTok{sk  =}\StringTok{ }\KeywordTok{import}\NormalTok{(}\StringTok{"sklearn.ensemble"}\NormalTok{)}
\NormalTok{skd =}\StringTok{ }\KeywordTok{import}\NormalTok{(}\StringTok{"sklearn.datasets"}\NormalTok{)}
\end{Highlighting}
\end{Shaded}

Définir le modèle RandomForest

\begin{Shaded}
\begin{Highlighting}[]
\NormalTok{RFpy  =}\StringTok{ }\NormalTok{sk}\OperatorTok{$}\KeywordTok{RandomForestClassifier}\NormalTok{(}\DataTypeTok{n\_estimators=}\NormalTok{2000L)}
\end{Highlighting}
\end{Shaded}

Appliquer le modèle aux données DNA

\begin{Shaded}
\begin{Highlighting}[]
\NormalTok{X         =}\StringTok{ }\KeywordTok{as.matrix}\NormalTok{(DNA[train, }\OperatorTok{{-}}\KeywordTok{ncol}\NormalTok{(DNA)])}
\NormalTok{Y         =}\StringTok{ }\NormalTok{np}\OperatorTok{$}\KeywordTok{array}\NormalTok{(DNA}\OperatorTok{$}\NormalTok{Class[train])}
\NormalTok{fitpy     =}\StringTok{ }\NormalTok{RFpy}\OperatorTok{$}\KeywordTok{fit}\NormalTok{(X, Y)}
\NormalTok{Xnew      =}\StringTok{ }\KeywordTok{as.matrix}\NormalTok{(DNA[test, }\OperatorTok{{-}}\KeywordTok{ncol}\NormalTok{(DNA)])}
\NormalTok{Ynew      =}\StringTok{ }\NormalTok{np}\OperatorTok{$}\KeywordTok{array}\NormalTok{(DNA}\OperatorTok{$}\NormalTok{Class[test])}
\NormalTok{pred\_py   =}\StringTok{ }\NormalTok{RFpy}\OperatorTok{$}\KeywordTok{predict}\NormalTok{(Xnew)}
\KeywordTok{adjustedRandIndex}\NormalTok{(pred\_py, Ynew)}
\end{Highlighting}
\end{Shaded}

\begin{verbatim}
## [1] 0.8654098
\end{verbatim}

\hypertarget{utiliser-une-fenetre-interactive-python-dans-la-session-r}{%
\subsubsection{3.2.2 Utiliser une fenêtre interactive Python dans la
session
R}\label{utiliser-une-fenetre-interactive-python-dans-la-session-r}}

\begin{Shaded}
\begin{Highlighting}[]
\KeywordTok{repl\_python}\NormalTok{()}
\NormalTok{import sklearn.ensemble}
\NormalTok{from sklearn.ensemble import RandomForestClassifier}
\NormalTok{clf =}\StringTok{ }\KeywordTok{RandomForestClassifier}\NormalTok{(}\DataTypeTok{n\_estimators=}\DecValTok{2000}\NormalTok{)}
\end{Highlighting}
\end{Shaded}

\hypertarget{importation-de-script-python}{%
\subsubsection{3.2.3 Importation de script
Python}\label{importation-de-script-python}}

Remarque : il est possible d'intégrer du code Python en Rmarkdown.

\begin{Shaded}
\begin{Highlighting}[]
\ImportTok{from}\NormalTok{ sklearn.ensemble }\ImportTok{import}\NormalTok{ RandomForestClassifier}
  \KeywordTok{def}\NormalTok{ RF\_DNA(X, y):}
\NormalTok{  clf }\OperatorTok{=}\NormalTok{ RandomForestClassifier(n\_estimators}\OperatorTok{=}\DecValTok{100}\NormalTok{)}
\NormalTok{  RF }\OperatorTok{=}\NormalTok{ clf.fit(X, y)}
  \ControlFlowTok{return}\NormalTok{ RF}
\end{Highlighting}
\end{Shaded}

Dans R, on importe le script écrit en Python :

\begin{Shaded}
\begin{Highlighting}[]
\KeywordTok{source\_python}\NormalTok{(}\StringTok{"RF\_DNA.py"}\NormalTok{)}
\NormalTok{pred =}\StringTok{ }\KeywordTok{RF\_DNA}\NormalTok{(X, y)}
\end{Highlighting}
\end{Shaded}


\end{document}
